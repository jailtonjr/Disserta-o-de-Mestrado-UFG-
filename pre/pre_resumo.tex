\chaves{Processo de Teste de Software, DevOps, Métodos Ágeis, MPT.Br}

\begin{resumo} 

\textbf{Contexto:} O mercado de Tecnologia da Informação (T.I.) é crescente. Na era da informação, as economias mundiais vêm investindo cada vez mais no mercado de Serviços. Dentro deste cenário competitivo, o teste de software é um importante componente para elevação da qualidade do software desenvolvido no Brasil e sua competitividade mundial. Contudo, as Micro e Pequenas Empresas (MPEs) possuem recursos limitados para investimentos em processos, ferramentas e modelos de maturidade de teste de software em seus negócios. Diante disto, este trabalho tem como proposta principal produzir um aparato para melhoria do processo de teste de software para MPEs. \textbf{Objetivo:} Como objetivos principais, este estudo propôs uma versão mais atualizada do processo FreeTest, bem como instruções práticas de como implantar as atividades sugeridas no processo, tudo isso formatado em um novo processo e um guia de implantação respectivamente. \textbf{Metodologia:} Com o propósito de cumprir os objetivos almejado neste trabalho foram criados o processo FreeTest 2.0 como uma melhoria do processo do Método FreeTest, focado principalmente em técnicas Ágeis, DevOps e alinhado ao ecossistema das MPEs. E o FreeTest Wizard, que consiste em um guia de implantação que apoia a implantação do processo de forma didática, dinâmica e flexível. Outra contribuição deste trabalho foi a criação de ferramentas de apoio para disseminação deste conhecimento e gestão dos conteúdos, neste caso a criação de uma plataforma web, distribuída de forma gratuita e no formato "\textit{as a Service}". Por fim, os \textbf{resultados} e \textbf{conclusões} poderão ser vistos no capítulo final deste trabalho.

\end{resumo}