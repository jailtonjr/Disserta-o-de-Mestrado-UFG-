\chapter{\textit{Case} de execução do FreeTest 2.0}
\label{sec:caseexecucaofreetest20}


\section{Case: MPE-PDV - Fictício}
\label{sec:apendicempe}

\subsection{A instituição}

Fundada em 1998 por seu proprietário João da Silva a MPE-PDV Sistemas hoje está consolidada como uma empresa líder no mercado de software para PDV (Ponto de Venda). Com sede em Goiânia - GO atende todo o Brasil com sua solução web voltada para Pontos de Venda, através do seu carro chefe o MPE-PDV, que obviamente leva o mesmo nome da empresa. Empresa com vinte funcionários, divididos dentro outros departamentos por suporte e desenvolvimento, que ocupam o maior quadro de pessoas da organização.

\subsection{O desafio}

A MPE-PDV é uma empresa como várias outras, possui um software carro chefe que é vendido sob um modelo de negócio do tipo “contrato de prestação de serviços”, no qual a fornecedora (MPE-PDV) “aluga” o software para seu cliente (consumidor), que por sua vez, paga um valor mensal que é determinado pelo o uso do software e pela quantidade de usuários que seu cliente possui para utilizar o sistema.

Por ser uma empresa comprometida em atender todos os seus mais de 150 clientes em todo o Brasil, a MPE-PDV de dois em dois meses gera novas versões do seu software, com o intuito de levar melhorias aos seus clientes e correções de \textit{bugs}, quando encontrados. Isso é importante, pois sempre que há uma mudança na lei que impacta seu software ou um de seus clientes solicita uma melhoria a MPE-PDV deve em tempo hábil desenvolvê-la, testá-la (ou não) sempre que possível.

Sua equipe de desenvolvimento e teste trabalha sempre a todo o vapor, em outras palavras, sempre apagando incêndios! A equipe de testes é composta por dois analistas, que realizam testes funcionais na aplicação e relatam os erros encontrados no próprio sistema de \textit{help desk}. O sistema de \textit{Help Desk} é onde todos as demandas, tanto de melhorias (externas ou internas), quanto correções são registradas. A quantidade de erros encontradas é relativamente grande, mas poderia ser maior... Uma das justificativas da equipe de testes é que além da \textit{“demora de se criar os ambientes ideais para se testar a aplicação, vira e mexe um build vem instável”}. Isso quer dizer, que a aplicação não executa corretamente, em outras palavras, é um grande transtorno, pois como a equipe de testes não tem acesso ou apoia a criação dos \textit{builds} não consegue detectar este tipo de problema de forma precoce, deste modo dependendo da ajuda de um desenvolvedor (que provavelmente está ocupado).

Outro grande problema relatado pela equipe de testes é que alguns erros triviais são encontrados, no entanto a questão é que muitos destes erros são impeditivos, isso quer dizer que é impossível testar novos cenários até a correção deste erro. Senão bastassem os desafios diários, o MPE-PDV é um software legado, com baixo nível de documentação, com inúmeras regras de negócio complexas e interdependentes e com uma arquitetura monolítica, que segundo os testadores dificultam ainda mais o seu trabalho.

Este cenário é cíclico e se repete a cada projeto para lançar uma nova versão. Como as demandas nunca param de chegar e os prazos são sempre curtos, muitas vezes é necessário realizar entregas do MPE-PDV sem a quantidade de testes necessárias, mas qual é a quantidade de testes necessária? Aí já é outra história...

\subsection{A solução}

Após a implantação do Método FreeTest 2.0 a MPE-PDV Sistemas obteve algumas melhorias. Por escolha da MPE-PDV o Freetest 2.0 foi implantado até o nível 4 de maturidade, a decisão foi feita pela equipe técnica envolvida na implantação do processo, o motivo foi que, segundo a equipe boa parte dos gargalos que ela enfrentavam poderiam ser resolvidos com as práticas especificas disponíveis no nível quatro de maturidade.

Agora graças as práticas de gerenciamento de projetos de teste (área de processo GPT) a equipe de testes está mais ativa no planejamento das atividades, pois consegue realizar análises de risco, definir estratégias, escopos do que testar e criar estimativas mais fieis ao que será executado, pois agora os próprios testadores estão aplicando seu \textit{know-how} para definir o planejamento de suas atividades. O mais importante é que, com essas práticas de gerenciamento do projeto de testes, a equipe de testes se tornou mais ativa nos projetos, conseguindo interagir desde concepção, isso tornou a equipe mais engajada, melhorou o trabalho em equipe e reduziu ruídos na comunicação.

Com as atividades planejadas, cronogramas bem definidos e todos sabendo o que devem fazer agora ficou mais fácil executar os testes funcionais, isso porque após adotar o FreeTest 2.0, novas ferramentas de apoio foram implantadas para ajudar a equipe de testes, entre elas o TestLink e MantisBT (área de processo ET). O Testlink ajuda na especificação e execução dos casos de teste e torna o trabalho em equipe mais dinâmico e transparente, pois qualquer um da organização consegue acompanhar as atividades de teste. O MantisBT, por outro lado, permite que os erros sejam reportados da forma devida, desta forma não sendo mais necessário usar a antiga ferramenta de Help Desk para reportar erros e melhorias, pois agora ela será somente de uso do suporte e dos clientes. Com uma ferramenta adequada para reportar os erros, identificação, especificação e execução dos casos de teste ficou mais fácil gerar dados que serão utilizados como métricas ao fim dos projetos, isso contribuirá para a melhoria contínua do processo na empresa.

Outro problema que era corriqueiro na MPE-PDV Sistemas era a grande quantidade de \textit{builds} falhas e erros inseridos. No entanto, com a ajuda de novas abordagens sugeridas pelo FreeTest 2.0 este número de problemas foi reduzido, pois agora a equipe de desenvolvimento com auxilio de ferramentas de gerência de configuração (área de processo GCT) e análise estática de código, tem feito a revisão de código fonte e aplicado \textit{style checkers} e \textit{bugs checkers} (área de processo AES), desta maneira, com a colaboração da equipe de desenvolvimento erros tem sido encontrados prematuramente. Um ganho que vale ressaltar foi com a aplicação de testes de regressão (área de processo TRG), pois pela própria natureza do software com uma grande quantidade de regras de negócios interdependentes, era normal que uma alteração pontual em uma região do software impactasse outra, contudo agora com a aplicação de testes de regressão e apoio das ferramentas de teste e futuramente automação dos casos de teste ficou mais fácil evidenciar estes \textit{bugs}. Estas práticas surtiram grande resultados na empresa e animou a equipe de suporte que esporadicamente ouvia de seus usuários reclamações como, “A funcionalidade XPTO que funcionava na versão anterior, não funciona mais!”.

Foi natural que com a implantação de ferramentas de gestão e controle dos testes, dados de execução, \textit{bugs} registrados e quantidade de re-testes feitos, tais dados se tornassem insumos para métricas (área de processo MED). Com isso a empresa pode coletar e analisar os dados e utilizá-los para melhoria contínua do processo, produto e até mesmo no planejamento estratégico da empresa! Com informações mais fidedignas, artefatos e um conjunto de ferramentas de apoio a empresa logo começou a automatizar vários funcionalidades do MPE-PDV, desta maneira, criando constantemente \textit{scripts} de teste para regressão (área de processo TRG) e testes de aceitação (área de processo TDA) que são executados dentro de uma ambiente totalmente integrado, graças a uma ferramenta de integração contínua (área de processo INC), também proposta pelo FreeTest 2.0, de agora em diante, não só os testes escritos pelos testadores são executados pela integração contínua, como também toda a geração de \textit{build} e \textit{deploy} é feita de forma automática, aumentando o feedback, evitando perca de tempo com criação do ambiente de testes e gerando rápidas respostas para corrigir eventuais falhas.

Agora com as práticas de teste bem definidas, métricas de todos os projetos e com o apoio de diversas ferramentas a empresa MPE-PDV já pensa em expandir sua operação e criar uma nova ferramenta no mesmo seguimento, contudo com um modelo de negócio \textit{As-a-Service} e para isso já estuda como irá desenvolver, testar e entregar seu software de forma contínua (área de processo TCA).