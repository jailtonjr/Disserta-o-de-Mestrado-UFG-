\chapter{Conclusão e Trabalhos Futuros}
\label{sec:conclusaoetrabalhosfuturos}

Este capítulo finaliza o estudo, apresentando a conclusão deste e os trabalhos futuros. Este capítulo está organizado da seguinte forma: a seção \ref{sec:finalconclusao} apresenta a conclusão do trabalho e a seção \ref{sec:trabalhosfuturos} apresenta os trabalhos futuros. 

\section{Conclusão}
\label{sec:finalconclusao}

Ao longo dos estudos realizados e do conhecimento empírico dos autores deste trabalho, principalmente com o enfoque em MPEs observou-se que as Organizações devido às suas restrições, principalmente financeira e de recursos humanos, desistem ou deixam em segundo plano uma política de qualidade de software na empresa. Tratando-se de um processo de testes, constatou-se que as limitações de recursos humanos com \textit{know-how} teórico e prático em processos de qualidade, conhecimento de em ferramentas de apoio e informação de como implantar uma dada atividades é um grande limitador, logo com a intenção de apoiar essa situação este trabalho propôs, a melhoria do processo de teste do Método FreeTest, criação de uma ferramenta para manutenção do processo escrito e modelado e um guia de implantação customizável que funciona como um \textit{wizard} de implantação do processo; todo o aparato ferramental e de conhecimento foi distribuído de forma gratuita e num formato \textit{SaaS}.

As melhorias propostas para o processo do Freetest serão de grande contribuição para a melhoria da maturidade das MPEss. A forma que a estrutura do processo foi definida, ou seja, espelhado em modelos de processos já conhecidos internacionalmente, contudo com um embasamento em metodologias ágeis, DevOps e outras boas práticas irão tornar os processos das organizações mais ágeis e com uma visão de automação de algumas tarefas de forma mais fácil e barata, pois a automação quando bem planejada e executada contribui na redução dos custos. 

Com o intuito de apoiar às organizações, foram desenvolvidos ferramentas de apoio à manutenção dos processos da empresa, entregue a comunidade no formato web, podendo ser acessado por qualquer um que tenha um cadastro na mesma. Além de possibilitar o acesso aos processos da organização é possível evoluir ou estender o processo do Método FreeTest, o que torna uma ferramenta flexível e aderente à organização. 

Por fim, com a ajuda do guia de implantação será possível que organizações, principalmente com o perfil de micro e pequenas empresas possam implantar o processo de teste de forma autônoma e prática com o apoio do \textit{wizard} de implantação, desta maneira, reduzindo custos com consultorias técnicas e com ferramentas de apoio pagas, já que o método proposto aconselha o uso de ferramentas de código aberto.

\section{Trabalhos Futuros}
\label{sec:trabalhosfuturos}

Esse trabalho abre um precedente no que diz respeito a entrega de um processo de teste de software como serviço. A cerca dos estudos realizados sobre automação estima-se que em outra fase será possível tornar tarefas automatizadas a partir do processo modelado na plataforma web, desta maneira podendo monitorar ou até mesmo executar determinadas tarefas via ambiente totalmente virtualizado. 

Acerca dos estudos realizados sobre DevOps, acredita-se que a criação e manutenção de ambientes de teste poderá ser mantida dentro da plataforma, transformando a infraestrutura como código (\textit{infraestructure as a code}) e utilizando a virtualização de ambientes de teste de maneira automáticas e de fácil acesso através da plataforma web. Será possível então reduzir custos com ambientes de trabalho e reduzir típicos falsos erros oriundos de ambientes, corroborando então com práticas de Entrega Contínua (\textit{Continuous Delivery}). 

Outra proposta de trabalho futuro é a incorporação de um Modelo de Competências relacionados ao teste de software. Buscando na literatura não foi encontrado nenhum processo de uso específico que utilize um Modelo de Competência para auxiliar as organizações a determinar com base nas competências o melhor papel para determinada área de atuação. Desta maneira como trabalhos futuros esperamos evoluir o método FreeTest com um Modelo de Competência baseado no SWECOM (\textit{Software Engineering Capability Model}) \cite{swecom2017} focado em micro e pequenas empresas. 