\chaves{Melhoria de Processo de Teste de Software, DevOps, Frameworks Ágeis, MPT.Br}

\begin{resumo} 
O mercado de Tecnologia da Informação (T.I.) é crescente. Na era da informação, as economias mundiais vêm investindo cada vez mais no mercado de Serviços. Dentro desse cenário competitivo, o teste de software é um importante componente para elevação da qualidade do software desenvolvido no Brasil e sua competitividade mundial. Contudo, as Micro e Pequenas Empresas (MPEs) e \textit{startups} possuem recursos limitados para investimentos em processos, ferramentas e modelos de maturidade de teste de software em seus negócios. Diante disso, este trabalho tem como proposta geral produzir um aparato para melhoria de processo de teste de software para MPEs e \textit{startups}. Dentro desse aparato foram criados um modelo de processo de teste genérico online e customizável, contendo lista de práticas especificas, níveis de maturidade e lista de ferramentas de apoio. Para facilitar a implantação do processo de teste, um guia de implantação foi criado para auxiliar as MPEs e \textit{startups} na implantação do processo de teste. Esse guia de implantação funcionará como um assistente de implantação (\textit{"wizard"}) online, que com base em uma série de perguntas, auxiliará a empresa a realizar a implantação do processo da melhor maneira possível, no ponto de vista de seu nível de maturidade e ferramentas de apoio que o processo suporta.

O processo de teste definido foi redesenhado com base no antigo processo de teste do Método FreeTest, neste caso agora, contendo áreas de processo, atividades específicas e ferramentas que estão alinhados ao ecossistema de MPEs e \textit{startups}. O enfoque desse novo processo foi inserir atividades passíveis de automação, visto que foram atividades bem-aceitas pelas MPEs, pois em seu contexto, geralmente a equipe técnica é formada em grande maioria por desenvolvedores e uma pequena parcela por \textit{testers}, facilitando assim a entrada de mecanismos automatizados para realização de atividades com enfoque na qualidade do código. Por fim, o guia de implantação e o \textit{framework} de processo foi disponibilizado em uma plataforma web e será utilizado para facilitar a implantação e manutenção do processo.

\end{resumo}

